% Preâmbulo: Configuração inicial do documento com pacotes compatíveis
\documentclass[12pt, a4paper]{article}
\usepackage[utf8]{inputenc}
\usepackage[T1]{fontenc}
\usepackage[portuguese]{babel}
\usepackage{geometry}
\usepackage{setspace}
\usepackage{indentfirst}
\usepackage{titlesec}
\usepackage{tocloft}
\usepackage{amsmath}
\usepackage{svg}
\usepackage{float}
\usepackage{graphicx}
\usepackage{listings}
\usepackage{xcolor}
\usepackage{hyperref}
\usepackage{enumitem}

% Configuração de margens conforme ABNT NBR 14724:2011
\geometry{left=3cm, right=2cm, top=3cm, bottom=2cm}

% Configuração de espaçamento e fonte
\onehalfspacing
\renewcommand{\familydefault}{\rmdefault}

% Configuração de títulos conforme ABNT
\titleformat{\section}{\normalfont\large\bfseries}{\thesection}{1em}{}
\titleformat{\subsection}{\normalfont\normalsize\bfseries}{\thesubsection}{1em}{}
\titleformat{\subsubsection}{\normalfont\normalsize\bfseries}{\thesubsubsection}{1em}{}

% Configuração do sumário
\renewcommand{\cftsecleader}{\cftdotfill{\cftdotsep}}
\renewcommand{\cftsecdotsep}{\cftdot}
\setlength{\cftbeforesecskip}{0.5em}
\setlength{\cftbeforesubsecskip}{0.2em}

% Configuração de hiperlinks
\hypersetup{
    colorlinks=true,
    linkcolor=black,
    urlcolor=blue,
    citecolor=black
}

% Configuração para listagens de código
\lstset{
    basicstyle=\ttfamily\small,
    breaklines=true,
    frame=single,
    numbers=left,
    numberstyle=\tiny,
    keywordstyle=\color{blue},
    stringstyle=\color{red},
    commentstyle=\color{gray},
    showstringspaces=false
}

\begin{document}

% --- Capa ---
\begin{titlepage}
    \centering
    \vspace*{1cm}
    {\large\bfseries CENTRO UNIVERSITÁRIO\\ CATÓLICA DE SANTA CATARINA\\
    POLO JOINVILLE\\
    CURSO DE ENGENHARIA DE SOFTWARE}\par
    \vspace{2cm}
    {\Large\bfseries FAZENDAPRO - GERENCIADOR DE PECUÁRIA COM FOCO EM LEITE}\par
    \vspace{2cm}
    {\normalsize\bfseries Gustavo Henrique Dias}\par
    \vspace{1cm}
    \vspace{3cm}
    {\normalsize JOINVILLE\\
    18 DE JUNHO DE 2025}\par
\end{titlepage}

% --- Sumário ---
\tableofcontents
\newpage

% --- Resumo ---
\section*{Resumo}
\addcontentsline{toc}{section}{Resumo}
\begin{spacing}{1.5}
O projeto FazendaPro é uma plataforma desenvolvida como projeto para otimizar a gestão de rebanhos bovinos, com ênfase na pecuária leiteira. O sistema propõe uma interface técnica e funcional para o gerenciamento integrado de animais, pastagens e produção de leite, fornecendo ferramentas para monitoramento detalhado e análise de dados. Suas principais funcionalidades incluem o registro do histórico genético, sanitário e produtivo do gado, com foco especial no acompanhamento de vacas em lactação para otimizar a produção leiteira e identificar anomalias. O projeto visa atender às necessidades de pequenos e médios produtores, promovendo eficiência operacional e rastreabilidade no setor agropecuário.
\end{spacing}
\vspace{0.5cm}
\textbf{Palavras-chave:} Gestão pecuária, Pecuária leiteira, Rastreabilidade, Produção de leite, Tecnologia agropecuária.
\newpage

% --- Introdução ---
\section{Introdução}

\subsection{Contexto}
\begin{spacing}{1.5}
O projeto FazendaPro é desenvolvido no âmbito da pecuária brasileira, com foco na gestão de rebanhos bovinos e ênfase na produção leiteira, um setor de relevância econômica que responde por cerca de 20\% do PIB agropecuário nacional \cite{agro20}. Apesar de sua importância, pequenos e médios produtores enfrentam desafios significativos, como a dependência de métodos manuais ou planilhas para gestão, resultando em ineficiências, falta de rastreabilidade do histórico genético e sanitário dos animais, dificuldades na previsão de produtividade leiteira e custos operacionais elevados. Segundo a Embrapa, Empresa Brasileira de Pesquisa Agropecuária,  aproximadamente 80\% dos produtores leiteiros brasileiros utilizam práticas tradicionais de baixa tecnologia, o que limita a competitividade e a eficiência \cite{anuario2023}. No norte de Minas Gerais, por exemplo, produtores relatam dificuldades em valorizar seus animais no mercado devido à ausência de registros detalhados, conforme noticiado pelo portal G1 \cite{g12022}.
\end{spacing}

\subsection{Justificativa}
\begin{spacing}{1.5}
O desenvolvimento do FazendaPro justifica-se pela necessidade de modernizar a gestão pecuária, com foco na pecuária leiteira, para pequenos e médios produtores que enfrentam barreiras no acesso à tecnologias avançadas. A pecuária leiteira é uma atividade econômica essencial, empregando cerca de 4 milhões de pessoas no Brasil \cite{4milhao}, mas a falta de ferramentas acessíveis compromete a eficiência e a competitividade. Um produtor do norte de Minas Gerais relatou perdas na comercialização de animais devido à ausência de registros detalhados sobre genética e saúde, conforme destacado em reportagem do G1 \cite{g12022}. Além disso, sistemas comerciais existentes frequentemente apresentam custos elevados e interfaces inadequadas às necessidades rurais, conforme apontado análise pelo blog Agrolink \cite{agropec2024}. O FazendaPro, como projeto acadêmico, propõe uma solução técnica, acessível e escalável, com ênfase na rastreabilidade de rebanhos e no monitoramento da produção leiteira, contribuindo para a eficiência operacional, a valorização do gado e a sustentabilidade do setor.
\end{spacing}

\subsection{Objetivos}
\begin{spacing}{1.5}
\subsubsection{Objetivo Principal}
Desenvolver uma plataforma digital para a gestão integrada de rebanhos bovinos, com foco na pecuária leiteira, que permita o registro detalhado do histórico genético, sanitário e produtivo dos animais, promovendo a rastreabilidade e a eficiência na produção de leite.

\subsubsection{Objetivos Secundários}
\begin{enumerate}[label=\alph*)]
    \item Desenvolver um sistema acessível e de baixo custo para pequenos e médios produtores rurais;
    \item Implementar funcionalidades para monitoramento em tempo real da produção de leite e da saúde animal;
    \item Automatizar processos operacionais, como notificações de prenhez e gerenciamento de lotes, reduzindo a dependência de tarefas manuais;
    \item Fornecer dashboards analíticos com dados de desempenho para apoiar a tomada de decisão;
    \item Habilitar a exportação de históricos em PDF para facilitar a comercialização de animais;
    \item Desenvolver uma interface responsiva e funcional, otimizada para dispositivos móveis.
\end{enumerate}

% --- Descrição do Projeto ---
\section{Descrição do Projeto}

\subsection{Tema do Projeto}

\begin{spacing}{1.5}
O projeto FazendaPro tem como objetivo desenvolver uma aplicação web para a gestão de rebanhos bovinos leiteiros, voltada especialmente para pequenos e médios produtores. A plataforma oferece uma solução acessível, baseada em tecnologia moderna, para apoiar a coleta e análise de dados sobre genética, saúde, alimentação, reprodução e produção de leite. Com isso, busca-se promover maior controle, rastreabilidade e eficiência na produção, alinhando-se à transformação digital do agronegócio.
\end{spacing}

\subsubsection{Problemas a Resolver}
\begin{enumerate}[label=\alph*)]
    \item \textbf{Falta de rastreabilidade}: ausência de registros estruturados sobre o histórico dos animais (nascimento, vacinação, produção etc.).
    \item \textbf{Baixa eficiência na gestão}: controle manual ou informal dificulta decisões baseadas em dados.
    \item \textbf{Falta de acessibilidade às tecnologias disponíveis}: falta de acessibilidade ferramentas simples e eficazes adaptadas ao ambiente rural\cite{uspDemocrata}.
\end{enumerate}

\subsection{Limitações}
\begin{spacing}{1.5}
O projeto FazendaPro apresenta as seguintes limitações, identificadas com base no escopo e nos requisitos definidos:

\begin{enumerate}[label=\alph*)]
    \item \textbf{Dependência de Conectividade à Internet}: A aplicação web requer conexão estável, o que pode ser um desafio em áreas rurais com acesso limitado à internet, impactando funcionalidades como monitoramento em tempo real e notificações via WhatsApp.
    \item \textbf{Foco Primário na Pecuária Leiteira}: Embora o sistema contemple a gestão geral de rebanhos bovinos, seu foco principal é a pecuária leiteira, limitando a aplicabilidade a outros tipos de pecuária ou atividades agrícolas.
    \item \textbf{Ausência de Funcionalidades Offline}: Não há suporte para operações sem conexão, o que pode dificultar o uso em áreas com conectividade instável.
    \item \textbf{Curva de Aprendizado para Usuários}: Produtores com baixa familiaridade tecnológica podem enfrentar dificuldades na adoção de funcionalidades analíticas.
    \item \textbf{Dependência de Serviços de Terceiros}: A utilização de Heroku, JawsDB e WhatsApp introduz riscos relacionados a custos e disponibilidade.
    \item \textbf{Escalabilidade Limitada}: O sistema foi projetado inicialmente para pequenos e médios produtores (até 10.000 animais), podendo necessitar re-arquitetura para operações de grande escala.
    \item \textbf{Custos Operacionais}: Dependência de serviços em nuvem pode gerar custos crescentes conforme o volume de dados e usuários, sem análise detalhada de ROI para diferentes cenários.
    \item \textbf{Integração WhatsApp}: Limitações da API não-oficial do WhatsApp podem impactar as notificações, com risco de bloqueio ou mudanças nas políticas de uso.
    \item \textbf{Conformidade Regulatória}: Embora contemple LGPD, pode necessitar adaptações para outras regulamentações internacionais em caso de expansão.
    \item \textbf{Tolerância a Falhas}: Ausência de redundância geográfica pode impactar a disponibilidade em caso de falhas regionais dos provedores de nuvem.
\end{enumerate}
\end{spacing}

% --- Especificação Técnica ---
\section{Especificação Técnica}

\subsection{Requisitos Funcionais (RF)}
\begin{spacing}{1.5}
\begin{enumerate}[label=RF0\arabic{*}.]
    \item \textbf{Acessar o Sistema}
    \begin{enumerate}[label=RF01.0\arabic{*}]
        \item O sistema deve permitir que o usuário faça o login na plataforma com suas credenciais, que são fornecidas pelos desenvolvedores do software.
        \item O sistema deve validar as credenciais do usuário e conceder acesso apenas para os usuários que geraram o token (\textit{JSON Web Token}).
    \end{enumerate}
    \item \textbf{Adicionar um Animal}
    \begin{enumerate}[label=RF02.0\arabic{*}]
        \item O sistema deve permitir que o usuário cadastre um novo animal no sistema.
        \item O sistema deve permitir incluir dados do animal como, no mínimo: identificação (nome e número do brinco), data de nascimento, genitora, filho (caso exista), raça, sexo e informações de saúde (vacinas).
    \end{enumerate}
    \item \textbf{Gerenciar o Animal}
    \begin{enumerate}[label=RF03.0\arabic{*}]
        \item O sistema deve permitir que o usuário edite ou exclua as informações de um animal já cadastrado.
        \item O sistema deve oferecer a opção de exportar o histórico do animal em formato PDF.
    \end{enumerate}
    \item \textbf{Inserir Informações do Animal}
    \begin{enumerate}[label=RF04.0\arabic{*}]
        \item O sistema deve permitir que o usuário insira informações adicionais sobre o animal, como registros de vacinas, alimentação, tratamentos ou eventos, como nascimento de filhotes.
    \end{enumerate}
    \item \textbf{Registrar Peso do Animal por Mês/Semana}
    \begin{enumerate}[label=RF05.0\arabic{*}]
        \item O sistema deve permitir que o usuário registre o peso do animal em intervalos regulares (mensal ou semanal).
        \item O sistema deve armazenar esses registros para acompanhamento do desenvolvimento do animal.
        \item O sistema deve permitir a edição ou exclusão desses registros.
    \end{enumerate}
    \item \textbf{Mudar de Lote}
    \begin{enumerate}[label=RF06.0\arabic{*}]
        \item O sistema deve mudar automaticamente o lote ao qual um animal pertence com base em sua produção de leite ou outros critérios configuráveis.
    \end{enumerate}
    \item \textbf{Definir Data de Prenhez}
    \begin{enumerate}[label=RF07.0\arabic{*}]
        \item O sistema deve permitir que o usuário registre a data de prenhez de uma vaca.
        \item O sistema deve notificar o usuário (via WhatsApp) quando a data de prenhez estiver próxima do parto, 20 dias antes.
    \end{enumerate}
    \item \textbf{Vender o Animal}
    \begin{enumerate}[label=RF8.0\arabic{*}]
        \item O sistema deve permitir que o usuário registre a venda de um animal.
        \item O sistema deve atualizar o status do animal para ``vendido'' e registrar a data da venda.
        \item O sistema deve oferecer a opção de exportar o histórico do animal em PDF no momento da venda.
        \item O sistema deve permitir verificar o histórico de todas as vendas dentro do módulo de vendas.
    \end{enumerate}
    \item \textbf{Cadastrar Vacinas}
    \begin{enumerate}[label=RF9.0\arabic{*}]
        \item O sistema deve permitir que o usuário cadastre vacinas para posterior vinculação aos animais.
        \item O sistema deve permitir a pesquisa de vacinas por datas.
    \end{enumerate}
    \item \textbf{Sair da Plataforma}
    \begin{enumerate}[label=RF10.0\arabic{*}]
        \item O sistema deve permitir que o usuário faça o logout da plataforma.
    \end{enumerate}
    \item \textbf{Relatórios Avançados}
    \begin{enumerate}[label=RF11.0\arabic{*}]
        \item O sistema deve gerar relatórios de produtividade por animal, lote e período.
        \item O sistema deve fornecer análises preditivas de produção de leite baseadas em histórico.
        \item O sistema deve gerar relatórios de eficiência reprodutiva e sanitária do rebanho.
        \item O sistema deve permitir exportação de relatórios em múltiplos formatos (PDF, Excel, CSV).
        \item O sistema deve oferecer dashboards interativos com filtros personalizáveis.
    \end{enumerate}
\end{enumerate}
\end{spacing}

\subsection{Requisitos Não Funcionais (RNF)}
\begin{spacing}{1.5}
\begin{enumerate}[label=RNF0\arabic{*}.]
    \item \textbf{Estilização}
    \begin{enumerate}[label=RNF01.0\arabic{*}]
        \item A estilização da aplicação deve seguir os padrões de estilo do Figma.
        \item Para facilitar a estilização, deve ser usado Tailwind ou outra biblioteca de CSS.
        \item Componentes padrões devem ser criados para seguir um padrão geral.
        \item As cores da aplicação devem apresentar-se de forma funcional e acessível.
    \end{enumerate}
    \item \textbf{Ferramentas}
    \begin{enumerate}[label=RNF02.0\arabic{*}]
        \item Para o frontend, deve-se utilizar React com bibliotecas para facilitar o fetch de informações.
        \item Para o backend, será usado NestJS para autenticação e notificações, enquanto Go será utilizado para as demais funcionalidades.
    \end{enumerate}
    \item \textbf{Idiomas}
    \begin{enumerate}[label=RNF03.0\arabic{*}]
        \item Todo o desenvolvimento deve respeitar variáveis de idioma.
        \item O idioma principal será PT-BR, com possibilidade de implementação futura de EN-US e ES-ES.
    \end{enumerate}
    \item \textbf{Separar Módulo}
    \begin{enumerate}[label=RF04.0\arabic{*}]
        \item O sistema deve organizar as informações em módulos acessíveis via menu lateral (Dashboard, Animais, Fornecedores, Vendas, Estoque).
    \end{enumerate}
    \item \textbf{Analisar Dashboards}
    \begin{enumerate}[label=RF05.0\arabic{*}]
        \item O sistema deve fornecer dashboards com informações analíticas sobre os animais, como produção de leite, saúde geral e tendências de desempenho.
    \end{enumerate}
    \item \textbf{Desempenho}
    \begin{enumerate}[label=RNF06.0\arabic{*}]
        \item O sistema deve ter tempo de resposta inferior a 2 segundos para operações básicas (CRUD).
        \item O sistema deve suportar até 100 usuários simultâneos sem degradação de performance.
        \item O sistema deve carregar a página inicial em menos de 3 segundos.
        \item O sistema deve processar relatórios complexos em menos de 10 segundos.
    \end{enumerate}
    \item \textbf{Escalabilidade}
    \begin{enumerate}[label=RNF07.0\arabic{*}]
        \item O sistema deve suportar até 10.000 animais por fazenda sem perda de performance.
        \item O sistema deve permitir expansão horizontal através de containers.
        \item O sistema deve implementar cache distribuído para otimizar consultas frequentes.
        \item O sistema deve suportar sharding de banco de dados para grandes volumes de dados.
    \end{enumerate}
    \item \textbf{Disponibilidade e Tolerância a Falhas}
    \begin{enumerate}[label=RNF08.0\arabic{*}]
        \item O sistema deve ter disponibilidade de 99.5\% mensal.
        \item O sistema deve implementar circuit breakers para serviços externos.
        \item O sistema deve ter plano de contingência para falhas de serviços terceiros.
        \item O sistema deve realizar backup automático diário dos dados.
    \end{enumerate}
    \item \textbf{Conectividade e Suporte Offline}
    \begin{enumerate}[label=RNF09.0\arabic{*}]
        \item O sistema deve funcionar com conexões de internet limitadas (3G).
        \item O sistema deve implementar cache local para operações básicas offline.
        \item O sistema deve sincronizar dados automaticamente quando a conexão for restabelecida.
        \item O sistema deve comprimir dados para reduzir uso de banda.
    \end{enumerate}
    \item \textbf{Otimizações Móveis}
    \begin{enumerate}[label=RNF10.0\arabic{*}]
        \item O sistema deve implementar lazy loading para imagens e componentes.
        \item O sistema deve comprimir assets CSS/JS em produção.
        \item O sistema deve ter interface responsiva para dispositivos móveis.
        \item O sistema deve implementar Progressive Web App (PWA) para melhor experiência móvel.
    \end{enumerate}
    \item \textbf{Conformidade e Privacidade}
    \begin{enumerate}[label=RNF11.0\arabic{*}]
        \item O sistema deve estar em conformidade com a LGPD para tratamento de dados.
        \item O sistema deve implementar anonimização de dados sensíveis em relatórios.
        \item O sistema deve manter logs de auditoria por no mínimo 5 anos.
        \item O sistema deve permitir exclusão completa de dados conforme direito ao esquecimento.
    \end{enumerate}
        \item \textbf{Conformidade com LGPD}
    \begin{enumerate}[label=RF12.0\arabic{*}]
        \item O sistema deve implementar mecanismos de consentimento explícito para coleta e uso de dados sensíveis dos animais.
        \item O sistema deve permitir que o usuário visualize, edite e exclua dados pessoais conforme direitos da LGPD.
        \item O sistema deve manter logs de auditoria para todas as operações com dados sensíveis.
        \item O sistema deve permitir a exportação de dados em formato legível conforme direito de portabilidade.
    \end{enumerate}
    \item \textbf{Integração com IoT}
    \begin{enumerate}[label=RF13.0\arabic{*}]
        \item O sistema deve permitir integração futura com dispositivos IoT para coleta automática de dados de produção de leite.
    \end{enumerate}
\end{enumerate}
\end{spacing}

\subsection{Diagrama de Casos de Uso}
\begin{spacing}{1.5}
O diagrama de casos de uso está apresentado na Figura \ref{fig:cases-of-use}. Nela busca-se explicar um pouca da história do usuário através das principais funcionalidade e ideais do sistema. Nesse exemplo, o usuário acessará o sistema e poderá gerenciar ou cadastrar o animal, independente da decisão ele terá que fornecer as informações do animal, como idade, raça, etc.
\begin{figure}[H]
    \centering
    \includegraphics[width=1\textwidth]{images/gerenciador-caso-de-uso.drawio.png}
    \caption{Diagrama de Casos de Uso - Gerenciador do Animal}
    \label{fig:cases-of-use}
\end{figure}
\end{spacing}
Poderá também fazer o registro do peso do animal já existente, e caso respeite a regra específica, o animal é trocado de lote por um worker.

\begin{figure}[H]
    \centering
    \includegraphics[width=1\textwidth]{images/movimentacao-caso-de-uso.drawio.png}
    \caption{Diagrama de Casos de Uso - Movimentação}
    \label{fig:cases-of-use}
\end{figure}

 O usuário poderá visualizar o histórico do animal e vendê-lo. A ação de vender implica na exportação de um PDF com as informações e histórico do animal, conforme é mostrado nas figuras 3 e 4.

\begin{figure}[H]
    \centering
    \includegraphics[width=1\textwidth]{images/venda-caso-de-uso.drawio.png}
    \caption{Diagrama de Casos de Uso - Venda}
    \label{fig:cases-of-use}
\end{figure}
\end{spacing}

\begin{figure}[H]
    \centering
    \includegraphics[width=1\textwidth]{images/relatorios-casos-de-uso.drawio.png}
    \caption{Diagrama de Casos de Uso - Relatórios}
    \label{fig:cases-of-use}
\end{figure}
\end{spacing}

\subsection{Diagrama de Classes}
\begin{spacing}{1.5}
O diagrama de classes está apresentado na Figura \ref{fig:classes-diagram}. Nele pode-se encontrar a fluxo das classes que será representado a nível de código. Ajuda o desenvolvedor a criar e visualizar as classes com maior facilidade.
\begin{figure}[H]
    \centering
    \includegraphics[width=1\textwidth]{images/classesDiagram-new.drawio.png}
    \caption{Diagrama de Classes}
    \label{fig:classes-diagram}
\end{figure}
\end{spacing}

\subsection{Modelo Lógico}
\begin{spacing}{1.5}
O modelo lógico está apresentado na Figura \ref{fig:entity-relationship}, e reproduzido abaixo. Com o modelo lógico é criada toda a regra de negócio. Ele se refere às tabelas do banco de dados.
\begin{figure}[H]
\centering
\includegraphics[width=1\textwidth]{images/modelo-logico.drawio-1.png}
\caption{Modelo Lógico}
\label{fig:entity-relationship}
\end{figure}
\end{spacing}

\subsection{Considerações de Design}
\subsubsection{Visão Inicial da Arquitetura}
\begin{spacing}{1.5}
O projeto adota uma arquitetura modular, equilibrando a simplicidade de um monolito com a flexibilidade de serviços. O NestJS facilita a organização em módulos, permitindo escalabilidade futura para extração de serviços, como notificações. Essa escolha, juntamento com Go, suporta a gestão eficiente de rebanhos, com foco na pecuária leiteira, mantendo a robustez para expansão.
\end{spacing}

\subsubsection{Padrões de Arquitetura}
\begin{spacing}{1.5}
O projeto utiliza uma arquitetura limpa baseada em DDD (Domain-Driven Design) com arquitetura hexagonal, garantindo separação de responsabilidades e manutenção facilitada.
\end{spacing}

\begin{lstlisting}[language=bash, caption={Estrutura de Diretórios do Projeto}]
fazendapro-api/
├── api/
│   ├── handlers/
│   │   ├── user.go
│   │   └── product.go
│   └── middleware/
│       └── auth.go
├── cmd/
│   └── app/
│       └── main.go
├── config/
│   └── config.go
├── internal/
│   ├── models/
│   │   ├── user.go
│   │   └── product.go
│   ├── repository/
│   │   ├── user_repository.go
│   │   └── product_repository.go
│   └── service/
│       ├── user_service.go
│       └── product_service.go
├── pkg/
│   └── jwt/
│       └── jwt.go
├── scripts/
├── tests/
│   ├── handlers/
│   ├── repository/
│   └── service/
├── .env
├── go.mod
├── go.sum
└── README.md
\end{lstlisting}

\subsubsection{Modelo C4 — Diagrama de Contexto, Containers e Componentes}
\begin{spacing}{1.5}
Na Figura \ref{fig:C4-Contexto} é apresentado o Diagrama de Contexto, ilustrando as interações do sistema FazendaPro com atores externos, como o usuário (produtor) e serviços de notificações (WhatsApp).

\begin{figure}[H]
    \centering
    \includegraphics[width=0.2\textwidth]{images/c1.drawio.png}
    \caption{Modelo C4 — Diagrama de Contexto do sistema FazendaPro}
    \label{fig:C4-Contexto}
\end{figure}

\vspace{0.5cm}

Na Figura \ref{fig:C4-Containers} é detalhado o Diagrama de Containers, descrevendo os componentes de software, como frontend, backend, banco de dados, cache, monitoramento e pipeline de deploy, com ênfase na interação com a gestão de rebanhos e produção leiteira.

\begin{figure}[H]
    \centering
    \includegraphics[width=0.8\textwidth]{images/c2.drawio.png}
    \caption{Modelo C4 — Diagrama de Containers do sistema FazendaPro}
    \label{fig:C4-Containers}
\end{figure}

\vspace{0.5cm}

Na Figura \ref{fig:C4-Componentes} é apresentado o Diagrama de Componentes, detalhando os módulos do API Server, incluindo autenticação, gerenciamento de animais, vacinas, lotes, prenhez, vendas e dashboards analíticos, com foco na rastreabilidade e produção leiteira.

\begin{figure}[H]
    \centering
    \includegraphics[width=1\textwidth]{images/c3.drawio.png}
    \caption{Modelo C4 — Diagrama de Componentes do sistema FazendaPro}
    \label{fig:C4-Componentes}
\end{figure}
\end{spacing}

\subsubsection{Aplicação Web}
\begin{spacing}{1.5}
A aplicação web será desenvolvida com React, enquanto o API Server, hospedado em container no Heroku, utilizará NestJS, Go com Redis (SSPL - Server Side Public License) para caching em memória, garantindo desempenho na gestão de dados pecuários.
\end{spacing}

\subsubsection{Armazenamento Persistente de Dados}
\begin{spacing}{1.5}
O armazenamento utiliza JawsDB. A aplicação web realiza requisições HTTP (REST) ao API Server, que consulta o JawsDB via conexão SQL e utiliza Redis para caching, otimizando o acesso a dados de rebanhos e produção leiteira.
\end{spacing}

\subsubsection{Arquitetura de Alta Disponibilidade e Escalabilidade}
\begin{spacing}{1.5}
\textbf{Estratégias de Escalabilidade:}
\begin{enumerate}[label=\alph*)]
\item \textbf{Escalabilidade Horizontal}: Arquitetura baseada em containers permite adicionar instâncias conforme demanda.
\item \textbf{Load Balancing}: Distribuição de carga automática via Heroku Load Balancer.
\item \textbf{Database Sharding}: Particionamento por fazenda para suportar até 10.000 animais por instância.
\item \textbf{Cache Distribuído}: Redis Cluster para cache compartilhado entre instâncias.
\item \textbf{CDN}: Cloudflare para entrega otimizada de assets estáticos.
\end{enumerate}

\textbf{Tolerância a Falhas:}
\begin{enumerate}[label=\alph*)]
\item \textbf{Circuit Breakers}: Implementação de circuit breakers para serviços externos (WhatsApp, JawsDB).
\item \textbf{Retry Logic}: Tentativas automáticas com backoff exponencial para falhas temporárias.
\item \textbf{Health Checks}: Monitoramento contínuo de saúde dos serviços com alertas automáticos.
\item \textbf{Graceful Degradation}: Funcionalidades essenciais continuam operando mesmo com falha de serviços secundários.
\item \textbf{Database Replication}: Réplicas de leitura para distribuir carga e garantir disponibilidade.
\end{enumerate}

\textbf{Plano de Contingência:}
\begin{enumerate}[label=\alph*)]
\item \textbf{Multi-Region Backup}: Backup automático em múltiplas regiões geográficas.
\item \textbf{Disaster Recovery}: RTO de 4 horas e RPO de 1 hora para restauração completa.
\item \textbf{Alternate Providers}: Configuração pré-estabelecida para migração rápida para AWS/GCP.
\item \textbf{Offline Mode}: Cache local para operações críticas durante indisponibilidade da internet.
\end{enumerate}

\textbf{Monitoramento e Observabilidade:}
\begin{enumerate}[label=\alph*)]
\item \textbf{Métricas de Performance}: Tempo de resposta, throughput, taxa de erro.
\item \textbf{Alertas Proativos}: Notificação automática quando métricas excedem thresholds.
\item \textbf{Distributed Tracing}: Rastreamento de requisições através de múltiplos serviços.
\item \textbf{Business Metrics}: Monitoramento de KPIs específicos do negócio (animais cadastrados, relatórios gerados).
\item \textbf{Capacity Planning}: Análise preditiva para planejamento de capacidade baseado em tendências de uso.
\end{enumerate}
\end{spacing}

\subsection{Stack Tecnológica}
\begin{spacing}{1.5}
\subsubsection{Linguagens de Programação}
As linguagens Go, TypeScript e JavaScript foram escolhidas para serem padrões no projeto por sua eficiência, tipagem estática e adequação ao desenvolvimento web.


\subsubsection{Frameworks e Bibliotecas}
\begin{enumerate}[label=\alph*)]
\item \textbf{React} – Biblioteca JavaScript para construção de interfaces de usuário.
\item \textbf{Nest.js} – Framework backend para Node.js com foco em arquitetura escalável.
\item \textbf{TypeORM} – ORM que facilita a manipulação de bancos de dados com TypeScript.
\item \textbf{Go} – Linguagem de programação eficiente e concorrente, usada para serviços rápidos.
\item \textbf{JWT} – Padrão de token para autenticação segura entre cliente e servidor.
\item \textbf{Bcrypt} – Biblioteca para criptografia de senhas.
\item \textbf{Express} – Framework minimalista para criação de APIs com Node.js.
\item \textbf{Styled Components} – Biblioteca para estilização de componentes React com CSS-in-JS.
\item \textbf{React Router} – Gerenciador de rotas em aplicações React.
\item \textbf{React Hook Form} – Biblioteca para gerenciamento eficiente de formulários em React.
\item \textbf{React Query} – Gerencia estados de dados assíncronos (como chamadas HTTP) em React.
\item \textbf{React Toastify} – Biblioteca para exibição de notificações (toasts) em React.
\item \textbf{React Icons} – Conjunto de ícones integrável ao React.
\item \textbf{Yup} – Biblioteca de validação de esquemas, usada com formulários.
\item \textbf{Jest} – Framework de testes para JavaScript/TypeScript.
\item \textbf{Cypress} – Ferramenta de testes end-to-end para aplicações web.
\item \textbf{Tailwind CSS} – Framework utilitário para estilização com classes CSS pré-definidas.
\end{enumerate}



\subsubsection{Ferramentas de Desenvolvimento e Gestão de Projeto}
\begin{enumerate}[label=\alph*)]
\item \textbf{Docker} – Plataforma de conteinerização para empacotar e executar aplicações.
\item \textbf{MySQL} – Sistema de gerenciamento de banco de dados relacional.
\item \textbf{Docker Compose} – Ferramenta para definir e executar múltiplos containers Docker.
\item \textbf{Figma} – Ferramenta colaborativa de design de interfaces e prototipação.
\item \textbf{GitHub Projects} – Ferramenta de gerenciamento de tarefas integrada ao GitHub.
\item \textbf{Heroku} – Plataforma em nuvem para deploy rápido de aplicações. Foi escolhida pela precificação de estudante.
\item \textbf{JawsDB} – Serviço de banco de dados MySQL baseado em nuvem, utilizado com Heroku.
\item \textbf{Redis} – Banco de dados em memória usado para cache e filas de mensagens.
\item \textbf{New Relic} – Plataforma de observabilidade e monitoramento de performance de apps.
\item \textbf{Sentry} – Ferramenta para monitoramento de erros e exceções em tempo real.
\item \textbf{Mermaid} – Linguagem de marcação para criação de diagramas em texto (usada com Markdown).
\end{enumerate}

\subsubsection{Justificativa da Stack Tecnológica}
\begin{spacing}{1.5}
A escolha das tecnologias foi baseada em critérios de performance, escalabilidade e custo-benefício:

\begin{enumerate}[label=\alph*)]
\item \textbf{Go}: Escolhido para serviços de alta performance devido à baixa latência (sub-milissegundo) e eficiência em processamento concorrente, essencial para lidar com até 10.000 animais.
\item \textbf{React + TypeScript}: Garante desenvolvimento ágil com tipagem estática, reduzindo bugs em produção e facilitando manutenção.
\item \textbf{NestJS}: Arquitetura modular que facilita escalabilidade horizontal e manutenção de código.
\item \textbf{Redis}: Cache em memória com latência de sub-milissegundo, otimizando consultas frequentes de dashboards.
\end{enumerate}
\end{spacing}

\subsubsection{Análise de Custos e Alternativas}
\begin{spacing}{1.5}
\textbf{Custos Estimados (mensal):}
\begin{enumerate}[label=\alph*)]
\item \textbf{Heroku}: \$7/mês (Hobby Dyno) + \$9/mês (Redis) = \$16/mês
\item \textbf{JawsDB}: \$9.99/mês (10GB MySQL)
\item \textbf{WhatsApp Business API}: \$0.005 por mensagem (estimativa 1000 mensagens/mês = \$5)
\item \textbf{Total estimado}: \$30.99/mês para até 100 usuários
\end{enumerate}

\textbf{Alternativas consideradas:}
\begin{enumerate}[label=\alph*)]
\item \textbf{AWS}: Maior complexidade, custo inicial similar, melhor escalabilidade
\item \textbf{DigitalOcean}: 50\% menor custo, maior complexidade de setup
\item \textbf{PostgreSQL local}: Menor custo operacional, maior complexidade de manutenção
\end{enumerate}

\textbf{Plano de Contingência:}
\begin{enumerate}[label=\alph*)]
\item Migração para AWS em caso de limitações do Heroku
\item Backup diário automático para múltiplas regiões
\item API de notificação alternativa (SMS/Email) em caso de problemas com WhatsApp
\end{enumerate}
\end{spacing}

\subsubsection{Integração com WhatsApp}
\begin{spacing}{1.5}
\textbf{Implementação técnica:}
\begin{enumerate}[label=\alph*)]
\item Utilização da biblioteca \texttt{whatsapp-web.js} para automação
\item Rate limiting de 20 mensagens por minuto conforme limitações da plataforma
\item Fallback para email em caso de falha na entrega via WhatsApp
\item Queue system com Redis para gerenciar envio de notificações
\end{enumerate}

\textbf{Limitações e riscos:}
\begin{enumerate}[label=\alph*)]
\item API não-oficial sujeita a bloqueios
\item Necessidade de QR Code scan periódico para manutenção da sessão
\item Limite de 256 conversas simultâneas
\item Política de uso do WhatsApp pode mudar
\end{enumerate}
\end{spacing}

\subsection{Considerações de Segurança}

\subsubsection{Conformidade com LGPD}
\begin{spacing}{1.5}
\begin{enumerate}[label=\alph*)]
    \item \textbf{Consentimento}: Implementação de mecanismo de opt-in explícito para coleta de dados sensíveis dos animais.
    \item \textbf{Direitos do titular}: Portal de autoatendimento para visualização, correção e exclusão de dados pessoais.
    \item \textbf{Minimização de dados}: Coleta apenas de dados estritamente necessários para a finalidade declarada.
    \item \textbf{Portabilidade}: Exportação de dados em formato estruturado (JSON/CSV) para facilitar migração.
    \item \textbf{Anonimização}: Técnicas de pseudonimização para dados em relatórios e analytics.
    \item \textbf{DPO}: Designação de Data Protection Officer para gestão de conformidade.
\end{enumerate}
\end{spacing}

\subsubsection{Autenticação e Autorização}
\begin{spacing}{1.5}
\begin{enumerate}[label=\alph*)]
    \item \textbf{Credenciais expostas}: Senhas armazenadas com hash Bcrypt (custo 12) e salt único.
    \item \textbf{Ataques de força bruta}: Rate limiting progressivo com @nestjs/throttler (3 tentativas/min).
    \item \textbf{JWT Security}: Tokens com expiração de 15 minutos, refresh tokens com rotação.
    \item \textbf{Multi-factor Authentication}: Implementação opcional via TOTP para usuários admin.
    \item \textbf{Session Management}: Invalidação automática de sessões inativas após 30 minutos.
\end{enumerate}
\end{spacing}

\subsubsection{Criptografia e Proteção de Dados}
\begin{spacing}{1.5}
\begin{enumerate}[label=\alph*)]
    \item \textbf{Dados em repouso}: Criptografia AES-256 para campos sensíveis no banco de dados.
    \item \textbf{Chaves de criptografia}: Gerenciamento via HSM ou AWS KMS para rotação automática.
    \item \textbf{Backup encryption}: Backups criptografados com chaves separadas do ambiente produtivo.
    \item \textbf{API Keys}: Armazenamento seguro de credenciais via variáveis de ambiente criptografadas.
\end{enumerate}
\end{spacing}

\subsubsection{Proteção contra Vulnerabilidades Web}
\begin{spacing}{1.5}
\begin{enumerate}[label=\alph*)]
    \item \textbf{SQL Injection}: Uso exclusivo de ORM com prepared statements, validação de entrada.
    \item \textbf{XSS}: Content Security Policy (CSP) rigoroso, sanitização de dados de entrada.
    \item \textbf{CSRF}: Tokens CSRF únicos por sessão, validação de origem (SameSite cookies).
    \item \textbf{Clickjacking}: Headers X-Frame-Options e CSP frame-ancestors.
    \item \textbf{SSRF}: Whitelist de URLs permitidas para integrações externas.
    \item \textbf{Input Validation}: Validação rigorosa com Yup/Joi em todas as entradas.
\end{enumerate}
\end{spacing}

\subsubsection{Logs de Auditoria e Monitoramento}
\begin{spacing}{1.5}
\begin{enumerate}[label=\alph*)]
    \item \textbf{Audit Trail}: Log completo de todas as operações com dados sensíveis (CRUD de animais).
    \item \textbf{Security Events}: Detecção e alerta de tentativas de acesso não autorizado.
    \item \textbf{Compliance Logs}: Armazenamento de logs por 5 anos conforme requisitos legais.
    \item \textbf{Log Integrity}: Assinatura digital de logs para prevenir adulteração.
    \item \textbf{SIEM Integration}: Integração com ferramentas de monitoramento (Sentry, New Relic).
    \item \textbf{Alertas automáticos}: Notificação para administradores em caso de atividades suspeitas.
\end{enumerate}
\end{spacing}

\subsubsection{Testes de Segurança}
\begin{spacing}{1.5}
\begin{enumerate}[label=\alph*)]
    \item \textbf{Penetration Testing}: Testes trimestrais por empresa especializada.
    \item \textbf{Vulnerability Scanning}: Varredura automatizada semanal com ferramentas como OWASP ZAP.
    \item \textbf{Dependency Scanning}: Verificação automática de vulnerabilidades em dependências (npm audit, Snyk).
    \item \textbf{Static Analysis}: Análise estática de código com SonarQube para identificar vulnerabilidades.
    \item \textbf{DAST}: Dynamic Application Security Testing em ambiente de staging.
\end{enumerate}
\end{spacing}

\subsubsection{Gestão de Incidentes de Segurança}
\begin{spacing}{1.5}
\begin{enumerate}[label=\alph*)]
    \item \textbf{Plano de Resposta}: Procedimento documentado para resposta a incidentes de segurança.
    \item \textbf{Notificação ANPD}: Processo para notificação à autoridade em até 72h conforme LGPD.
    \item \textbf{Comunicação}: Templates para comunicação com usuários afetados em caso de vazamento.
    \item \textbf{Forensics}: Ferramentas e procedimentos para análise forense de incidentes.
    \item \textbf{Business Continuity}: Plano de continuidade de negócios em caso de comprometimento.
\end{enumerate}
\end{spacing}

\subsection{Gitflow e Branches}
\begin{spacing}{1.5}
O projeto utiliza seis branches principais para ambientes de stage e produção, cobrindo serviços, backend e frontend:

\textbf{Stage:}
\begin{enumerate}[label=\alph*)]
    \item \texttt{back/develop} - backend.
    \item \texttt{front/develop} - frontend.
    \item \texttt{service/develop} - serviços.
\end{enumerate}

\textbf{Produção:}
\begin{enumerate}[label=\alph*)]
    \item \texttt{back/release} - backend.
    \item \texttt{front/release} - frontend.
    \item \texttt{service/release} - serviços.
\end{enumerate}

O fluxo de desenvolvimento envolve criar branches a partir da produção para novas funcionalidades, abrir Pull Requests para a release, testar em stage e, se aprovado, realizar o merge na produção.

% --- Próximos Passos ---
\section{Próximos Passos}
\begin{spacing}{1.5}
Os próximos passos incluem a implementação das funcionalidades descritas, com foco na construção do backend e frontend, testes de integração e validação em ambiente de stage. O cronograma será detalhado em documentos complementares, priorizando a entrega das funcionalidades principais até o final do primeiro semestre de 2025.
\end{spacing}

\appendix
\section{Apêndice} % Or 'Links Úteis', 'Recursos Online', etc.

\begin{enumerate}

    \item GO. \textit{Go Programming Language}. Disponível em: \url{https://go.dev}. Acesso em: 17 jun. 2025.
    \item TYPESCRIPT. \textit{The TypeScript Handbook}. Disponível em: \url{https://www.typescriptlang.org}. Acesso em: 17 jun. 2025.
    \item JAVASCRIPT. \textit{MDN Web Docs}. Disponível em: \url{https://developer.mozilla.org/pt-BR/docs/Web/JavaScript}. Acesso em: 17 jun. 2025.
    \item REACT. \textit{React Documentation}. Disponível em: \url{https://react.dev}. Acesso em: 17 jun. 2025.
    \item NESTJS. \textit{NestJS Documentation}. Disponível em: \url{https://nestjs.com}. Acesso em: 17 jun. 2025.
    \item TYPEORM. \textit{TypeORM Documentation}. Disponível em: \url{https://typeorm.io}. Acesso em: 17 jun. 2025.
    \item JWT. \textit{JSON Web Tokens}. Disponível em: \url{https://jwt.io}. Acesso em: 17 jun. 2025.
    \item BCRYPT. \textit{Bcrypt Documentation}. Disponível em: \url{https://www.npmjs.com/package/bcrypt}. Acesso em: 17 jun. 2025.
    \item EXPRESS. \textit{Express Documentation}. Disponível em: \url{https://expressjs.com}. Acesso em: 17 jun. 2025.
    \item STYLED COMPONENTS. \textit{Styled Components Documentation}. Disponível em: \url{https://styled-components.com}. Acesso em: 17 jun. 2025.
    \item REACT ROUTER. \textit{React Router Documentation}. Disponível em: \url{https://reactrouter.com}. Acesso em: 17 jun. 2025.
    \item REACT HOOK FORM. \textit{React Hook Form Documentation}. Disponível em: \url{https://react-hook-form.com}. Acesso em: 17 jun. 2025.
    \item REACT QUERY. \textit{React Query Documentation}. Disponível em: \url{https://tanstack.com/query}. Acesso em: 17 jun. 2025.
    \item REACT TOASTIFY. \textit{React Toastify Documentation}. Disponível em: \url{https://fkhadra.github.io/react-toastify/}. Acesso em: 17 jun. 2025.
    \item REACT ICONS. \textit{React Icons Documentation}. Disponível em: \url{https://react-icons.github.io/react-icons/}. Acesso em: 17 jun. 2025.
    \item YUP. \textit{Yup Documentation}. Disponível em: \url{https://github.com/jquense/yup}. Acesso em: 17 jun. 2025.
    \item JEST. \textit{Jest Documentation}. Disponível em: \url{https://jestjs.io}. Acesso em: 17 jun. 2025.
    \item SSPL. \textit{Server Side Public License
}. Disponível em: \url{https://www.mongodb.com/legal/licensing/server-side-public-license}. Acesso em: 3 jun. 2025.
    \item CYPRESS. \textit{Cypress Documentation}. Disponível em: \url{https://www.cypress.io}. Acesso em: 17 jun. 2025.
    \item TAILWIND CSS. \textit{Tailwind CSS Documentation}. Disponível em: \url{https://tailwindcss.com}. Acesso em: 17 jun. 2025.
    \item DOCKER. \textit{Docker Documentation}. Disponível em: \url{https://www.docker.com}. Acesso em: 17 jun. 2025.
    \item MYSQL. \textit{MySQL Documentation}. Disponível em: \url{https://www.mysql.com}. Acesso em: 17 jun. 2025.
    \item DOCKER COMPOSE. \textit{Docker Compose Documentation}. Disponível em: \url{https://docs.docker.com/compose/}. Acesso em: 17 jun. 2025.
    \item FIGMA. \textit{Figma Documentation}. Disponível em: \url{https://www.figma.com}. Acesso em: 17 jun. 2025.
    \item GITHUB PROJECTS. \textit{FazendaPro Project}. Disponível em: \url{https://github.com/orgs/fazendapro/projects/1}. Acesso em: 17 jun. 2025.
    \item HEROKU. \textit{Heroku Documentation}. Disponível em: \url{https://www.heroku.com}. Acesso em: 17 jun. 2025.
    \item JAWSDB. \textit{JawsDB Documentation}. Disponível em: \url{https://devcenter.heroku.com/articles/jawsdb}. Acesso em: 17 jun. 2025.
    \item REDIS. \textit{Redis Documentation}. Disponível em: \url{https://redis.io}. Acesso em: 17 jun. 2025.
    \item NEW RELIC. \textit{New Relic Documentation}. Disponível em: \url{https://newrelic.com}. Acesso em: 17 jun. 2025.
    \item SENTRY. \textit{Sentry Documentation}. Disponível em: \url{https://sentry.io}. Acesso em: 17 jun. 2025.
    \item MERMAID. \textit{Mermaid Documentation}. Disponível em: \url{https://mermaid.js.org}. Acesso em: 17 jun. 2025.
    \item C4 MODEL. \textit{C4 Model Documentation}. Disponível em: \url{https://c4model.com}. Acesso em: 17 jun. 2025.
    \item NESTJS THROTTLER. \textit{NestJS Throttler Documentation}. Disponível em: \url{https://docs.nestjs.com/techniques/throttler}. Acesso em: 17 jun. 2025.
    \item WHATSAPP WEB JS. \textit{WhatsApp Web.js Documentation}. Disponível em: \url{https://wwebjs.dev}. Acesso em: 17 jun. 2025.
    \item OWASP ZAP. \textit{OWASP Zed Attack Proxy}. Disponível em: \url{https://www.zaproxy.org}. Acesso em: 17 jun. 2025.
    \item SONARQUBE. \textit{SonarQube Documentation}. Disponível em: \url{https://www.sonarqube.org}. Acesso em: 17 jun. 2025.
    \item SNYK. \textit{Snyk Developer Security Platform}. Disponível em: \url{https://snyk.io}. Acesso em: 17 jun. 2025.
    \item CLOUDFLARE. \textit{Cloudflare CDN Documentation}. Disponível em: \url{https://www.cloudflare.com}. Acesso em: 17 jun. 2025.

\end{enumerate}


\begin{thebibliography}{99}

\bibitem{agro20}
CNN Brasil \textit{Agronegócio Brasileiro: entre narrativas e números}. Disponível em: \url{https://www.cnnbrasil.com.br/forum-opiniao/agronegocio-brasileiro-entre-narrativas-e-numeros/}

\bibitem{anuario2023}
EMBRAPA. \textit{Anuário Leite 2023: leite baixo carbono}. Juiz de Fora: Embrapa Gado de Leite, 2023. Disponível em: \url{https://www.embrapa.br/busca-de-publicacoes/-/publicacao/1154264/anuario-leite-2023-leite-baixo-carbono}. Acesso em: 24 jun. 2025.

\bibitem{g12022}
G1. \textit{Produtores de leite do norte de Minas enfrentam dificuldades com falta de rastreabilidade}. Disponível em: \url{https://globoplay.globo.com/v/2153861/}. Acesso em: 24 jun. 2025.

\bibitem{agropec2024}
Agrolink. \textit{Tecnologia na pecuária custa chegar ao produtor}. Disponível em: \url{https://www.agrolink.com.br/noticias/tecnologia-na-pecuaria-custa-chegar-ao-produtor_48656.html?utm_source=chatgpt.com}. Acesso em: 24 jun. 2025.

\bibitem{4milhao}
Agrolink \textit{Agronegócio Brasileiro: entre narrativas e números}. Disponível em: \url{https://www.agrolink.com.br/noticias/setor-leiteiro-emprega-cerca-de-4-milhoes-de-pessoas_465448.html}

\bibitem{uspDemocrata}
Agência FAPESP \text{Existem muitas tecnologias para o campo, mas falta democratizar o acesso, apontam especialistas}. Disponível em: \url{https://agencia.fapesp.br/existem-muitas-tecnologias-para-o-campo-mas-falta-democratizar-o-acesso-apontam-especialistas/44714}

\bibitem{lgpd2018}
BRASIL. Lei nº 13.709, de 14 de agosto de 2018. \textit{Lei Geral de Proteção de Dados Pessoais (LGPD)}. Brasília: Presidência da República, 2018. Disponível em: \url{http://www.planalto.gov.br/ccivil_03/_ato2015-2018/2018/lei/l13709.htm}. Acesso em: 24 jun. 2025.

\bibitem{anpd2024}
ANPD - Autoridade Nacional de Proteção de Dados. \textit{Guia de boas práticas para tratamento de dados pessoais}. Brasília: ANPD, 2024. Disponível em: \url{https://www.gov.br/anpd/pt-br}. Acesso em: 24 jun. 2025.

\end{thebibliography}

\clearpage

% --- Avaliações de Professores ---
\section{Avaliações de Professores}
\begin{spacing}{1.5}
\textbf{Considerações do Professor/a:}

\vspace{5cm}

\rule{0.5\textwidth}{0.4pt}

\vspace{1cm}

\textbf{Considerações do Professor/a:}

\vspace{5cm}

\rule{0.5\textwidth}{0.4pt}

\vspace{1cm}

\textbf{Considerações do Professor/a:}

\vspace{5cm}

\rule{0.5\textwidth}{0.4pt}
\end{spacing}

\end{document}