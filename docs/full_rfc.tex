\UseRawInputEncoding
% Preâmbulo: Configuração inicial do documento com pacotes compatíveis
\documentclass[12pt, a4paper]{article}
\usepackage[utf8]{inputenc}
\usepackage[T1]{fontenc}
\usepackage[portuguese]{babel}
\usepackage{geometry}
\usepackage{setspace}
\usepackage{indentfirst}
\usepackage{titlesec}
\usepackage{tocloft}
\usepackage{amsmath}
\usepackage{graphicx}
\usepackage{listings}
\usepackage{xcolor}
\usepackage{hyperref}
\usepackage{enumitem}

% Configuração de margens conforme ABNT NBR 14724:2011
\geometry{left=3cm, right=2cm, top=3cm, bottom=2cm}

% Configuração de espaçamento e fonte
\onehalfspacing
\renewcommand{\familydefault}{\rmdefault}

% Configuração de títulos conforme ABNT
\titleformat{\section}{\normalfont\large\bfseries}{\thesection}{1em}{}
\titleformat{\subsection}{\normalfont\normalsize\bfseries}{\thesubsection}{1em}{}
\titleformat{\subsubsection}{\normalfont\normalsize\bfseries}{\thesubsubsection}{1em}{}

% Configuração do sumário
\renewcommand{\cftsecleader}{\cftdotfill{\cftdotsep}}
\renewcommand{\cftsecdotsep}{\cftdot}
\setlength{\cftbeforesecskip}{0.5em}
\setlength{\cftbeforesubsecskip}{0.2em}

% Configuração de hiperlinks
\hypersetup{
    colorlinks=true,
    linkcolor=black,
    urlcolor=blue,
    citecolor=black
}

% Configuração para listagens de código
\lstset{
    basicstyle=\ttfamily\small,
    breaklines=true,
    frame=single,
    numbers=left,
    numberstyle=\tiny,
    keywordstyle=\color{blue},
    stringstyle=\color{red},
    commentstyle=\color{gray},
    showstringspaces=false
}

\begin{document}

% --- Capa ---
\begin{titlepage}
    \centering
    \vspace*{1cm}
    {\large\bfseries UNIVERSIDADE [NOME DA INSTITUIÇÃO]\\
    FACULDADE DE [NOME DA FACULDADE]\\
    CURSO DE ENGENHARIA DE SOFTWARE}\par
    \vspace{2cm}
    {\Large\bfseries FAZENDAPRO - SOLUÇÕES AGROPECUÁRIAS}\par
    \vspace{2cm}
    {\normalsize\bfseries Autor: Gustavo Henrique Dias}\par
    \vspace{1cm}
    {\normalsize Orientador: [Nome do Orientador]}\par
    \vspace{3cm}
    {\normalsize [CIDADE]\\
    17 DE JUNHO DE 2025}\par
\end{titlepage}

% --- Sumário ---
\tableofcontents
\newpage

% --- Resumo ---
\section*{Resumo}
\addcontentsline{toc}{section}{Resumo}
\begin{spacing}{1.5}
O projeto FazendaPro é uma solução agropecuária que visa facilitar a gestão de fazendas e a produção de leite. O sistema oferece uma interface intuitiva para gerenciar animais, pastagens e produção de leite, além de fornecer \textit{insights} para tomada de decisão. Uma das principais funcionalidades é a monitoração de vacas em lactação, permitindo acompanhar a produção de leite e identificar possíveis problemas, assim como manter seu histórico, como genitora, filho, etc.
\end{spacing}
\vspace{0.5cm}
\textbf{Palavras-chave:} Gestão agropecuária, Pecuária leiteira, Tecnologia, Rastreabilidade, Produção de leite.
\newpage

% --- Introdução ---
\section{Introdução}

\subsection{Contexto}
\begin{spacing}{1.5}
O projeto FazendaPro insere-se no contexto da pecuária leiteira brasileira, um setor essencial para a economia, mas que enfrenta desafios significativos, especialmente entre pequenos e médios produtores rurais. No Brasil, a pecuária leiteira é marcada por práticas tradicionais de gestão, com muitos fazendeiros utilizando métodos manuais ou planilhas para controlar animais, pastagens e produção de leite. Esse cenário resulta em ineficiências, como a falta de rastreamento do histórico genético e de saúde dos animais, dificuldades na previsão de faturamento e altos custos operacionais. O problema reside, principalmente, na ausência de ferramentas acessíveis que integrem dados de forma prática e ofereçam \textit{insights} para otimizar a produção. No norte de Minas Gerais, por exemplo, produtores relatam dificuldades em valorizar seus animais no mercado devido à falta de documentação detalhada sobre sua procedência e desempenho. O FazendaPro busca resolver essas dores, proporcionando uma solução tecnológica de baixo custo, adaptada às necessidades dos produtores rurais.
\end{spacing}

\subsection{Justificativa}
\begin{spacing}{1.5}
A criação do FazendaPro é justificada pela necessidade de modernizar a gestão pecuária, especialmente para pequenos e médios produtores que enfrentam barreiras no acesso a tecnologias avançadas. No Brasil, a pecuária leiteira representa uma fonte de renda para milhares de famílias, mas a falta de ferramentas acessíveis limita a competitividade e a lucratividade. Um fazendeiro do norte de Minas Gerais, por exemplo, relatou perdas significativas na venda de animais devido à ausência de um histórico detalhado que comprove sua qualidade genética e sanitária. Além disso, sistemas existentes no mercado muitas vezes possuem custos elevados, interfaces complexas e não atendem às necessidades específicas de produtores rurais. O FazendaPro é importante porque oferece uma solução de baixo custo, intuitiva e escalável, que permite o acompanhamento detalhado de cada animal, aumenta a valorização do gado no mercado e promove a tomada de decisões baseadas em dados. Ao democratizar o acesso à tecnologia, o projeto contribui para a sustentabilidade econômica e social do setor agropecuário, reduzindo custos operacionais, aumentando a eficiência e fortalecendo a rastreabilidade.
\end{spacing}

\subsection{Objetivos}
\begin{spacing}{1.5}
\subsubsection{Objetivo Principal}
Desenvolver uma plataforma digital que permita a gestão eficiente de fazendas leiteiras, garantindo a valorização do gado no mercado por meio do registro detalhado de seu histórico genético, sanitário e produtivo.

\subsubsection{Objetivos Secundários}
\begin{itemize}
    \item Criar um sistema de baixo custo que seja acessível a pequenos e médios produtores rurais.
    \item Implementar funcionalidades para monitoramento em tempo real da produção de leite e da saúde animal.
    \item Automatizar processos como notificações de prenhez e mudança de lotes, reduzindo o trabalho manual.
    \item Fornecer \textit{dashboards} analíticos que auxiliem na tomada de decisão com base em dados de desempenho.
    \item Garantir a exportação de históricos em PDF para facilitar negociações e vendas de animais.
    \item Desenvolver uma interface responsiva e intuitiva, adaptada para uso em dispositivos móveis.
\end{itemize}
\end{spacing}

% --- Descrição do Projeto ---
\section{Descrição do Projeto}

\subsection{Tema do Projeto}
\begin{spacing}{1.5}
O tema do projeto é o desenvolvimento de uma aplicação web focada na gestão de fazendas de gado leiteiro, com ênfase no registro histórico dos animais e na otimização da produção de leite. A FazendaPro combina tecnologia de ponta, como interfaces modernas e análises de dados, com uma abordagem acessível para atender às demandas de produtores rurais que enfrentam dificuldades na gestão manual ou com ferramentas de alto custo. O projeto abrange o controle de informações sobre o gado (genética, vacinas, alimentação, reprodução), a automação de processos operacionais e a geração de relatórios que aumentam a competitividade no mercado. Esse tema é relevante no contexto da transformação digital do agronegócio, promovendo eficiência, rastreabilidade e sustentabilidade.
\end{spacing}

\subsection{Problemas a Resolver}
\begin{spacing}{1.5}
O principal problema a ser resolvido é a garantia da valorização de um gado no mercado, por meio do seu histórico, desde o nascimento, genética, vacinas, alimentação, entre outras informações. Além de oferecer um sistema de baixo custo para produtores e fazendeiros que não têm acesso a tecnologias semelhantes por causa dos altos preços dos \textit{softwares} existentes no mercado.
\end{spacing}

\subsection{Limitações}
\begin{spacing}{1.5}
[Delimitação dos problemas que o projeto não abordará. Como não foi especificado no documento original, este campo pode ser preenchido pelo autor posteriormente.]
\end{spacing}

% --- Especificação Técnica ---
\section{Especificação Técnica}

\subsection{Requisitos Funcionais (RF)}
\begin{spacing}{1.5}
\begin{enumerate}[label=RF\arabic{*}.]
    \item \textbf{Acessar o Sistema}
    \begin{enumerate}[label=RF\arabic{*}.0\arabic{*}]
        \item O sistema deve permitir que o usuário faça o \textit{login} na plataforma com suas credenciais.
        \item O sistema deve validar as credenciais do usuário e conceder acesso apenas para os usuários que geraram o \textit{token}.
    \end{enumerate}
    \item \textbf{Adicionar um Animal}
    \begin{enumerate}[label=RF\arabic{*}.0\arabic{*}]
        \item O sistema deve permitir que o usuário cadastre um novo animal no sistema.
        \item O sistema deve permitir incluir dados do animal como, no mínimo: identificação (nome e número do brinco), data de nascimento, genitora, filho (caso exista), raça, sexo e informações de saúde (vacinas).
    \end{enumerate}
    \item \textbf{Gerenciar o Animal}
    \begin{enumerate}[label=RF\arabic{*}.0\arabic{*}]
        \item O sistema deve permitir que o usuário edite ou exclua as informações de um animal já cadastrado.
        \item O sistema deve oferecer a opção de exportar o histórico do animal em formato PDF.
    \end{enumerate}
    \item \textbf{Analisar Dashboards}
    \begin{enumerate}[label=RF05.0\arabic{*}]
        \item O sistema deve fornecer \textit{dashboards} com informações analíticas sobre os animais, como produção de leite, saúde geral e tendências de desempenho.
    \end{enumerate}
    \item \textbf{Inserir Informações do Animal}
    \begin{enumerate}[label=RF06.0\arabic{*}]
        \item O sistema deve permitir que o usuário insira informações adicionais sobre o animal, como registros de vacinas, alimentação, tratamentos ou eventos, como nascimento de filhotes.
    \end{enumerate}
    \item \textbf{Registrar Peso do Animal por Mês/Semana}
    \begin{enumerate}[label=RF07.0\arabic{*}]
        \item O sistema deve permitir que o usuário registre o peso do animal em intervalos regulares (mensal ou semanal).
        \item O sistema deve armazenar esses registros para acompanhamento do desenvolvimento do animal.
        \item O sistema deve permitir a edição ou exclusão desses registros.
    \end{enumerate}
    \item \textbf{Mudar de Lote}
    \begin{enumerate}[label=RF08.0\arabic{*}]
        \item O sistema deve mudar automaticamente o lote ao qual um animal pertence dependendo da sua produção de leite.
    \end{enumerate}
    \item \textbf{Definir Data de Prenhez}
    \begin{enumerate}[label=RF09.0\arabic{*}]
        \item O sistema deve permitir que o usuário registre a data de prenhez de uma vaca.
        \item O sistema deve notificar o usuário (via WhatsApp) quando a data de prenhez estiver próxima do parto, 20 dias antes.
    \end{enumerate}
    \item \textbf{Vender o Animal}
    \begin{enumerate}[label=RF10.0\arabic{*}]
        \item O sistema deve permitir que o usuário registre a venda de um animal.
        \item O sistema deve atualizar o status do animal para ``vendido'' e registrar a data da venda.
        \item O sistema deve oferecer a opção de exportar o histórico do animal em PDF no momento da venda.
        \item O sistema deve permitir verificar o histórico de todas as vendas dentro do módulo de vendas.
    \end{enumerate}
    \item \textbf{Cadastrar Vacinas}
    \begin{enumerate}[label=RF11.0\arabic{*}]
        \item O sistema deve permitir que o usuário cadastre a vacina para que depois ela seja vinculada ao animal.
        \item O sistema deve permitir a pesquisa de vacinas por datas.
    \end{enumerate}
    \item \textbf{Separar Módulo}
    \begin{enumerate}[label=RF12.0\arabic{*}]
        \item O sistema deve organizar as informações através de módulos dentro de um menu lateral (Dashboard, Vacas, Fornecedores, Vendas, Estoque).
    \end{enumerate}
    \item \textbf{Sair da Plataforma}
    \begin{enumerate}[label=RF13.0\arabic{*}]
        \item O sistema deve permitir que o usuário faça o \textit{logout} da plataforma.
    \end{enumerate}
\end{enumerate}
\end{spacing}

\subsection{Requisitos Não Funcionais (RNF)}
\begin{spacing}{1.5}
\begin{enumerate}[label=RNF\arabic{*}.]
    \item \textbf{Estilização}
    \begin{enumerate}[label=RNF\arabic{*}.0\arabic{*}]
        \item A estilização da aplicação deve seguir os padrões de estilo do Figma.
        \item Para facilitar a estilização, deve ser usado Tailwind ou outra biblioteca de CSS.
        \item Componentes padrões devem ser criados para seguir um padrão geral.
        \item As cores da aplicação devem apresentar-se de forma agradável.
    \end{enumerate}
    \item \textbf{Ferramentas}
    \begin{enumerate}[label=RNF\arabic{*}.0\arabic{*}]
        \item Para o \textit{Frontend}, deve-se utilizar React com bibliotecas para facilitar o \textit{fetch} das informações.
        \item Para o \textit{Backend}, será usado NestJS para serviços de autenticação e notificações, contudo, para todo o resto, será usado Go.
    \end{enumerate}
    \item \textbf{Idiomas}
    \begin{enumerate}[label=RNF\arabic{*}.0\arabic{*}]
        \item Todo o desenvolvimento deve ser feito respeitando variáveis de idioma.
        \item O idioma principal será PT-BR, posteriormente pode ser implementado EN-US e ES-ES.
    \end{enumerate}
    \item \textbf{Mobile}
    \begin{enumerate}[label=RNF04.0\arabic{*}]
        \item O desenvolvimento deve respeitar os casos de \textit{mobile}, respeitando um design responsivo e agradável.
    \end{enumerate}
\end{enumerate}
\end{spacing}

\subsection{Considerações de Design}

\subsubsection{Visão Inicial da Arquitetura}
\begin{spacing}{1.5}
Foi decidido usar a arquitetura modular para o projeto. A arquitetura modular oferece um equilíbrio entre a simplicidade de um monolito e a flexibilidade dos microsserviços. O NestJS facilita essa abordagem através de seus módulos bem definidos, permitindo escalabilidade sem a complexidade inicial dos microsserviços. Esta escolha permite que o sistema cresça naturalmente, com a possibilidade de extrair módulos para microsserviços no futuro (como é o caso das notificações no futuro).
\end{spacing}

\subsubsection{Padrões de Arquitetura}
\begin{spacing}{1.5}
A ideia seria usar uma arquitetura limpa como o DDD (\textit{Domain-Driven Design}) com arquitetura hexagonal.
\end{spacing}

\begin{lstlisting}[language=bash, caption={Estrutura de Diretórios do Projeto}]
fazendapro-api/
├── api/                    # Definições de handlers e endpoints HTTP
│   ├── handlers/           # Funções que lidam com requisições HTTP
│   │   ├── user.go         # Ex.: handler para rotas de usuário
│   │   └── product.go      # Ex.: handler para rotas de produtos
│   └── middleware/         # Middlewares (ex.: autenticação JWT)
│       └── auth.go         # Middleware de validação de token
├── cmd/                    # Ponto de entrada do projeto
│   └── app/                # Aplicação principal
│       └── main.go         # Arquivo principal que inicializa o servidor
├── config/                 # Configurações (ex.: variáveis de ambiente)
│   └── config.go           # Carrega .env ou outras configs
├── internal/               # Código interno (não exposto para outros pacotes)
│   ├── models/             # Estruturas de dados (ex.: User, Product)
│   │   ├── user.go
│   │   └── product.go
│   ├── repository/         # Acesso a dados (ex.: banco de dados)
│   │   ├── user_repository.go
│   │   └── product_repository.go
│   └── service/            # Lógica de negócio
│       ├── user_service.go
│       └── product_service.go
├── pkg/                    # Código reutilizável (se necessário)
│   └── jwt/                # Funções utilitárias para JWT
│       └── jwt.go
├── scripts/                # Scripts úteis (ex.: para build ou deploy)
├── tests/                  # Testes unitários e de integração
│   ├── handlers/
│   ├── repository/
│   └── service/
├── .env                    # Variáveis de ambiente (ex.: JWT_SECRET)
├── go.mod                  # Definição do módulo Go
├── go.sum                  # Dependências
└── README.md               # Documentação do projeto
\end{lstlisting}

\subsubsection{Modelo C4}
\begin{spacing}{1.5}
Os diagramas C4 (Contexto, Contêineres, Componentes) estão disponíveis no Apêndice A.
\end{spacing}

\subsubsection{Aplicação Web}
\begin{spacing}{1.5}
A aplicação web será desenvolvida com React. O \textit{API Server} será um servidor NestJS em \textit{container} no Heroku, funcionando como núcleo do sistema, incluindo Redis rodando no mesmo \textit{container} para \textit{caching} em memória.
\end{spacing}

\subsubsection{Armazenamento Persistente de Dados}
\begin{spacing}{1.5}
O armazenamento persistente será realizado com MySQL (JawsDB). A aplicação web faz requisições HTTP (REST) ao \textit{API Server}, que usa Redis internamente para \textit{caching} e consulta o JawsDB MySQL via conexão SQL.
\end{spacing}

\subsection{Stack Tecnológica}
\begin{spacing}{1.5}
\subsubsection{Linguagens de Programação}
As linguagens escolhidas incluem Go, TypeScript e JavaScript, justificadas pela sua eficiência, tipagem estática e ampla adoção em desenvolvimento web.

\subsubsection{Frameworks e Bibliotecas}
\begin{spacing}{1.5}
\begin{itemize}
    \item React
    \item Nest.js
    \item TypeORM
    \item Go
    \item JWT
    \item Bcrypt
    \item Express
    \item Styled Components
    \item React Router
    \item React Hook Form
    \item React Query
    \item React Toastify
    \item React Icons
    \item Yup
    \item Jest
    \item Cypress
    \item Tailwind CSS
\end{itemize}
\end{spacing}

\subsubsection{Ferramentas de Desenvolvimento e Gestão de Projeto}
\begin{itemize}
    \item Docker
    \item MySQL
    \item Docker Compose
    \item Figma
    \item GitHub Projects
    \item Heroku
    \item JawsDB
    \item Redis
    \item New Relic
    \item Sentry
    \item Mermaid
\end{itemize}
\end{spacing}

\subsection{Considerações de Segurança}

\subsubsection{Autenticação e Autorização}
\begin{spacing}{1.5}
\begin{itemize}
    \item \textbf{Credenciais expostas}: Será utilizado \textit{hash} para senhas com Bcrypt.
    \item \textbf{Ataques de força bruta}: Será usado limite de tentativas de \textit{login} (\textit{rate limiting}) com \texttt{@nestjs/throttler}.
\end{itemize}
\end{spacing}

\subsubsection{Exposição de Dados Sensíveis}
\begin{spacing}{1.5}
\begin{itemize}
    \item \textbf{Vazamento em respostas da API}: Será usado DTO para retornar apenas o necessário.
    \item \textbf{CORS}: Configurações adequadas para evitar acesso não autorizado.
    \item \textbf{Logs}: O Heroku usa HTTPS automaticamente para criptografar a comunicação.
\end{itemize}
\end{spacing}

\subsubsection{Injeção de Código}
\begin{spacing}{1.5}
\begin{itemize}
    \item \textbf{SQL Injection}: Será usado TypeORM, evitando \textit{queries} brutas.
    \item \textbf{XSS}: Será implementado Content Security Policy (CSP) no \textit{frontend}.
\end{itemize}
\end{spacing}

\subsection{Branches e Pipeline}
\begin{spacing}{1.5}
Atualmente, existem seis \textit{branches} principais relacionadas aos ambientes de \textit{stage} e produção, responsáveis por todos os componentes da aplicação: Serviços, \textit{Backend} e \textit{Frontend}.

\textbf{Stage:}
\begin{itemize}
    \item \texttt{back/develop} - para o \textit{backend}.
    \item \texttt{front/develop} - para o \textit{frontend}.
    \item \texttt{service/release} - para os microsserviços.
\end{itemize}

\textbf{Produção:}
\begin{itemize}
    \item \texttt{back/release} - para o \textit{backend}.
    \item \texttt{front/release} - para o \textit{frontend}.
    \item \texttt{service/release} - para os microsserviços.
\end{itemize}

O fluxo de desenvolvimento segue o seguinte processo: para cada nova \textit{feature}, cria-se uma \textit{branch} a partir da produção. Para testes, abre-se uma \textit{Pull Request} apontando para a \textit{release}, sobe-se para \textit{stage} e realizam-se os testes necessários. Caso os testes sejam aprovados, realiza-se o \textit{merge} na produção. O fluxo é representado no Apêndice B.
\end{spacing}

% --- Próximos Passos ---
\section{Próximos Passos}
\begin{spacing}{1.5}
Os próximos passos incluem a implementação das funcionalidades descritas, com foco na construção do \textit{backend} e \textit{frontend}, testes de integração e validação em ambiente de \textit{stage}. O cronograma para os Portfólios I e II será detalhado em documentos complementares, priorizando a entrega das funcionalidades principais até o final do primeiro semestre de 2026.
\end{spacing}

% --- Referências ---
\section{Referências}
\begin{spacing}{1.5}
\begin{itemize}
    \item GO. \textit{Go Programming Language}. Disponível em: \url{https://go.dev}. Acesso em: 17 jun. 2025.
    \item TYPESCRIPT. \textit{The TypeScript Handbook}. Disponível em: \url{https://www.typescriptlang.org}. Acesso em: 17 jun. 2025.
    \item JAVASCRIPT. \textit{MDN Web Docs}. Disponível em: \url{https://developer.mozilla.org/pt-BR/docs/Web/JavaScript}. Acesso em: 17 jun. 2025.
    \item REACT. \textit{React Documentation}. Disponível em: \url{https://react.dev}. Acesso em: 17 jun. 2025.
    \item NESTJS. \textit{NestJS Documentation}. Disponível em: \url{https://nestjs.com}. Acesso em: 17 jun. 2025.
    \item TYPEORM. \textit{TypeORM Documentation}. Disponível em: \url{https://typeorm.io}. Acesso em: 17 jun. 2025.
    \item JWT. \textit{JSON Web Tokens}. Disponível em: \url{https://jwt.io}. Acesso em: 17 jun. 2025.
    \item BCRYPT. \textit{Bcrypt Documentation}. Disponível em: \url{https://www.npmjs.com/package/bcrypt}. Acesso em: 17 jun. 2025.
    \item EXPRESS. \textit{Express Documentation}. Disponível em: \url{https://expressjs.com}. Acesso em: 17 jun. 2025.
    \item STYLED COMPONENTS. \textit{Styled Components Documentation}. Disponível em: \url{https://styled-components.com}. Acesso em: 17 jun. 2025.
    \item REACT ROUTER. \textit{React Router Documentation}. Disponível em: \url{https://reactrouter.com}. Acesso em: 17 jun. 2025.
    \item REACT HOOK FORM. \textit{React Hook Form Documentation}. Disponível em: \url{https://react-hook-form.com}. Acesso em: 17 jun. 2025.
    \item REACT QUERY. \textit{React Query Documentation}. Disponível em: \url{https://tanstack.com/query}. Acesso em: 17 jun. 2025.
    \item REACT TOASTIFY. \textit{React Toastify Documentation}. Disponível em: \url{https://fkhadra.github.io/react-toastify/}. Acesso em: 17 jun. 2025.
    \item REACT ICONS. \textit{React Icons Documentation}. Disponível em: \url{https://react-icons.github.io/react-icons/}. Acesso em: 17 jun. 2025.
    \item YUP. \textit{Yup Documentation}. Disponível em: \url{https://github.com/jquense/yup}. Acesso em: 17 jun. 2025.
    \item JEST. \textit{Jest Documentation}. Disponível em: \url{https://jestjs.io}. Acesso em: 17 jun. 2025.
    \item CYPRESS. \textit{Cypress Documentation}. Disponível em: \url{https://www.cypress.io}. Acesso em: 17 jun. 2025.
    \item TAILWIND CSS. \textit{Tailwind CSS Documentation}. Disponível em: \url{https://tailwindcss.com}. Acesso em: 17 jun. 2025.
    \item DOCKER. \textit{Docker Documentation}. Disponível em: \url{https://www.docker.com}. Acesso em: 17 jun. 2025.
    \item MYSQL. \textit{MySQL Documentation}. Disponível em: \url{https://www.mysql.com}. Acesso em: 17 jun. 2025.
    \item DOCKER COMPOSE. \textit{Docker Compose Documentation}. Disponível em: \url{https://docs.docker.com/compose/}. Acesso em: 17 jun. 2025.
    \item FIGMA. \textit{Figma Documentation}. Disponível em: \url{https://www.figma.com}. Acesso em: 17 jun. 2025.
    \item GITHUB PROJECTS. \textit{FazendaPro Project}. Disponível em: \url{https://github.com/orgs/fazendapro/projects/1}. Acesso em: 17 jun. 2025.
    \item HEROKU. \textit{Heroku Documentation}. Disponível em: \url{https://www.heroku.com}. Acesso em: 17 jun. 2025.
    \item JAWSDB. \textit{JawsDB Documentation}. Disponível em: \url{https://devcenter.heroku.com/articles/jawsdb}. Acesso em: 17 jun. 2025.
    \item REDIS. \textit{Redis Documentation}. Disponível em: \url{https://redis.io}. Acesso em: 17 jun. 2025.
    \item NEW RELIC. \textit{New Relic Documentation}. Disponível em: \url{https://newrelic.com}. Acesso em: 17 jun. 2025.
    \item SENTRY. \textit{Sentry Documentation}. Disponível em: \url{https://sentry.io}. Acesso em: 17 jun. 2025.
    \item MERMAID. \textit{Mermaid Documentation}. Disponível em: \url{https://mermaid.js.org}. Acesso em: 17 jun. 2025.
    \item C4 MODEL. \textit{C4 Model Documentation}. Disponível em: \url{https://c4model.com}. Acesso em: 17 jun. 2025.
\end{itemize}
\end{spacing}

% --- Apêndices ---
\section{Apêndices}
\begin{spacing}{1.5}
\textbf{Apêndice A - Diagramas C4}

Os diagramas de Contexto, Contêineres e Componentes (Modelo C4) estão disponíveis nos arquivos originais do projeto (\texttt{images/cases-of-use.drawio.svg}, \texttt{images/classesDiagram.drawio.png}, \texttt{images/entityRelationshipDiagram.drawio.png}).

\textbf{Apêndice B - Fluxo de Pipeline}

O diagrama do fluxo de \textit{pipeline} (Mermaid) está disponível no arquivo original do projeto.
\end{spacing}

% --- Avaliações de Professores ---
\section{Avaliações de Professores}
\begin{spacing}{1.5}
\textbf{Considerações do Professor/a:}

\vspace{5cm}

\rule{0.5\textwidth}{0.4pt}

\vspace{1cm}

\textbf{Considerações do Professor/a:}

\vspace{5cm}

\rule{0.5\textwidth}{0.4pt}

\vspace{1cm}

\textbf{Considerações do Professor/a:}

\vspace{5cm}

\rule{0.5\textwidth}{0.4pt}
\end{spacing}

\end{document}